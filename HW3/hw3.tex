\documentclass[11pt]{report}

\usepackage[utf8]{inputenc}
\usepackage[english]{babel}
\usepackage{amsfonts,amsmath,amsthm,amssymb,fullpage}

\title{CMPS 130: HW 3}
\author{Kevin Wang}

\newcounter{problem}
\setcounter{problem}{1}

\theoremstyle{definition}
\newtheorem{definition}{Definition}[problem]

\theoremstyle{plain}
\newtheorem{lemma}{Lemma}[problem]

\theoremstyle{plain}
\newtheorem{theorem*}{Theorem}
\newtheorem{theorem}{Theorem}[problem]
      
\begin{document}
\maketitle

\section*{Solution to Problem 1}

\begin{definition}[Kleene Closure]
$\Sigma^{*}$ is the set of all finite strings over $\Sigma$ 
as defined by $\underset{k \geq 0}{\bigcup} \Sigma^{k}$.\footnote{lec2.pdf}
\end{definition}

\begin{definition} [Recursively Enumerable Languages]
A language $\mathcal{L}$ is Recursively Enumerable (RE) if there exists some Turing Machine $M$,
such that $L(M)=\mathcal{L}$.\footnote{lec12.pdf}
\end{definition}

\hrule

\begin{theorem*}
Recursively Enumerable (RE) languages are closed under Kleene Closure.
\end{theorem*}

\begin{proof}
Let $\mathcal{L}$ be some RE language and let $M$ be some Turing Machine (TM) such that $L(M)=\mathcal{L}$ (Definition 1.2).
Let $\mathcal{L}^{*}$ be the Kleene Closure of $\mathcal{L}$. \newline

\noindent We define a TM $M^{*}$ that receives input $x$.
$M^{*}$ then non-deterministically splits $x$ into $s_{1},s_{2},\cdots,s_{k}$ where $k \leq |x|$.
Note that the number of ways to split $x$ is finite due to the length of input $x$.
Input $x$ is then accepted if for all $i \leq k$, $s_{i}$ is accepted.
The trivial case, $x= \epsilon$, is also accepted. \newline

\noindent Observe that $M^{*}$ simulates the Kleene Closure (Definition 1.1) and therefore accepts $\mathcal{L}^{*}$,
such that $L(M^{*})=\mathcal{L}^{*}$. Thus, $\mathcal{L}^{*}$ is Recursively Enumerable (Definition 1.2) --
proving that RE languages are closed under Kleene Closure.
\end{proof}

\pagebreak

\stepcounter{problem}
\section*{Solution to Problem 2}

\begin{theorem*}

\end{theorem*}

\begin{proof}

\end{proof}

\pagebreak

\stepcounter{problem}
\section*{Solution to Problem 3}

\begin{theorem*}

\end{theorem*}

\begin{proof}

\end{proof}

\pagebreak

\stepcounter{problem}
\section*{Solution to Problem 4}

\begin{theorem*}

\end{theorem*}

\begin{proof}

\end{proof}

\pagebreak

\stepcounter{problem}
\section*{Solution to Problem 5}

\begin{theorem*}

\end{theorem*}

\begin{proof}

\end{proof}

\pagebreak

\stepcounter{problem}
\section*{Solution to Problem 6}

\begin{theorem*}

\end{theorem*}

\begin{proof}

\end{proof}

\end{document}