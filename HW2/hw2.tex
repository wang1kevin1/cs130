\documentclass[11pt]{report}

\usepackage[utf8]{inputenc}
\usepackage[english]{babel}
\usepackage{amsfonts,amsmath,amsthm,amssymb,fullpage}

\title{CMPS 130: HW 2}
\author{Kevin Wang}

\newcounter{problem}
\setcounter{problem}{1}

\theoremstyle{definition}
\newtheorem{definition}{Definition}[problem]

\theoremstyle{plain}
\newtheorem{lemma}{Lemma}[problem]

\theoremstyle{plain}
\newtheorem{theorem*}{Theorem}
\newtheorem{theorem}{Theorem}[problem]
      
\begin{document}
\maketitle

\section*{Solution to Problem 1}

\begin{definition}
A language $\mathcal{L}$ is a context-free language (CFL) if there exists some CFG $G$, such that $L(G)=\mathcal{L}$.
\footnote{lec7.pdf}
\end{definition}

\hrule

\begin{theorem*}
All regular languages are context-free languages.
\end{theorem*}

\begin{proof}
Let DFA $D=(Q,\Sigma,\delta,q_{0},F)$ accept some regular language, $\mathcal{L}$, such that $L(D)=\mathcal{L}$. \newline

\noindent Let PDA $P=(Q,\Sigma,\Gamma,\delta', q_{0},\$,F)$ be constructed from DFA $D$, where
\begin{center}
$\delta : Q \times \Sigma \cup \{ \epsilon \} \times \Gamma \longrightarrow 2^{Q \times (\Gamma^{*} \cup \{ \epsilon \})}$ \\
$\delta (q,a,\epsilon) = (q', \epsilon)$
\end{center}
\noindent such that the stack of $P$ is not used. Therefore, $P$ is functionally equivalent to DFA $D$ 
and accepts language $\mathcal{L}$, such that $L(P)=\mathcal{L}$. \newline

\noindent By Definition 1.1, $\mathcal{L}$ is a context-free language. Thus, all regular languages are CFL's.
\end{proof}

\pagebreak

\stepcounter{problem}
\section*{Solution to Problem 2}

\begin{definition}
A language $\mathcal{L}$ is a context-free language (CFL) if there exists some CFG $G$, such that $L(G)=\mathcal{L}$.
\footnote{lec7.pdf}
\end{definition}

\begin{lemma}
Every regular language is a context-free language. \footnote{lec7.pdf}
\end{lemma}

\begin{lemma} [Closure Under Union]
Given CFL's $\mathcal{A}$ and $\mathcal{B}$, then $\mathcal{A} \cup \mathcal{B}$ is also a context-free language.
\footnote{Wikipedia, Context-free-language, Closure}
\end{lemma}
\hrule

\begin{theorem*}
The language $\mathcal{L}=\{ x| x \text{ is not of the form } ww \text{, } w \in \{0,1\}^{*} \}$ is a CFL.
\end{theorem*}

\begin{proof}
Let $\mathcal{L}=\mathcal{L}_{odd} \cup \mathcal{L}_{even}$ where:
\begin{enumerate}
\item $\mathcal{L}_{odd} = \{ x | x \in \mathcal{L}, |x| \text{ is odd} \}$
\item $\mathcal{L}_{even} = \{ xy | xy \in \mathcal{L}, |x| = |y|, x \neq y \}$
\end{enumerate}

\noindent We first observe that $\mathcal{L}$ accepts all words of odd length, as words of form $ww$ have even lengths.
We know that $\mathcal{L}_{odd}$ is a regular language\footnote{DFA from lec2.pdf}. 
By Lemma 2.1, $\mathcal{L}_{odd}$ is a CFL. \newline

\noindent Next, we construct a CFG $G=(V,\Sigma,P,S)$ that accepts a language $L(G)$:

\begin{enumerate}
\item $V=\{S,X,Y\}$
\item $\Sigma = \{0,1\}$
\item $P = \{X \longrightarrow 0X0|0X1|1X0|1X1|0$, $Y \longrightarrow 0Y0|0Y1|1Y0|1Y1|1\}$
\item $S = XY|YX$
\end{enumerate}

\noindent We note that for each production, $|X| = |Y|$, and thus $|XY|$ must be even.
Furthermore, because $X= \cdots|0$ and $Y= \cdots|1$, 
there is always at least a single symbol difference between $X$ and $Y$, and thus, $X \neq Y$.
Therefore, by construction of $G$, $L(G)=\mathcal{L}_{even}$. 
By Definition 2.1, we know that $\mathcal{L}_{even}$ is a CFL. \newline

\noindent Thus, because context-free languages are closed under union (Lemma 2.2), $\mathcal{L}$ is also a CFL.

\noindent 
\pagebreak
\end{proof}

\stepcounter{problem}
\section*{Solution to Problem 3}

\begin{lemma}
Let $M_{N}$ be a non-deterministic Turing Machine. 
There exists a deterministic Turing Machine $M_{D}$ that simulates $M_{N}$,
such that $L(M_{D})=L(M_{N})$.\footnote{lec10.pdf}
\end{lemma}

\hrule

\begin{theorem*}
A Turing Machine with an extra bit in the transition function adds no additional power.
\end{theorem*}

\begin{proof}
Observe that with an extra bit in the transition function, each state--symbol pair has 2 possible transitions.
The 2 transitions depend on new bit $D$ being $RIGHT$ or $LEFT/$\textvisiblespace. 
As such, the extension results in a non-deterministic Turing Machine 
as there is more than one explicit transition for each state-symbol pair.
Note that there is still a finite number of computational transitions.%
Let NTM $N$'s computational tree be represented by a binary tree with root $q_{0}$ and leaf nodes $f \in F$.
Left branches are transition functions $\delta(q_{i},\rho,LEFT)$ 
while right branches are transition functions $\delta(q_{i},\rho,RIGHT)$.\newline

\noindent We can construct a deterministic Turing Machine to simulate $N$ by using Breadth-First Search
on $N$'s computational tree. We start with 4 tapes:
\begin{enumerate}
\item Memory Tape: stores input string
\item Address Tape: stores the transitions of $N$'s computational tree
\item Direction Tape: stores direction the head moved the last time it visited the cell
\item Scratch Tape: used for running the simulation of $N$
\end{enumerate}

\noindent The Turing Machine then runs the simulation. 
First, the input string is read from memory into scratch.
For each iteration of the simulation, we read $D$ from the parallel cell from the direction tape.
We then run BFS to find a valid transition that accepts the state--symbol pair as well as $D$.
Then, we update the direction value to $D'$. \newline

\noindent As there exists a TM that simulates $N$, the extended Turing Machine is not more powerful.
\end{proof}

\pagebreak

\stepcounter{problem}
\section*{Solution to Problem 4}

\begin{theorem}
A Turing Machine with a 2-dimensional tape has the same power as a standard Turing Machine.
\end{theorem}

\begin{proof}
First, we will construct a Turing Machine with 2 tapes:\footnote{lec10.pdf}

\begin{enumerate}
\item Memory Tape: stores rows of symbols separated by \# and is bounded by \$ on either end
\item Counter Tape: tracks the column of the head by adding/removing markers every move R/L
\end{enumerate} 

\noindent Note that a Turing Machine with 2 tapes is equivalent in power to a standard Turing Machine. \newline

\noindent When the 2D Turing Machine moves LEFT, we move left on the memory tape. 
If the cell is at the leftmost boundary of the 2D table, 
the standard TM will have read in \$ or \#. The head then moves right by one step, 
returning to the initial symbol of the row. \newline

\noindent When the 2D Turing Machine moves RIGHT, we move right on the memory tape.
If the cell is at the current rightmost boundary of the 2D table,
the standard TM will have read in \$ or \#. 
A blank, \textvisiblespace, is subsequently inserted by 
shifting the symbol -- and everything right of it -- right by one cell.
This is equivalent to adding a blank at the end of a row in the 2D Turing Machine's table. \newline

\noindent When the 2D Turing Machine moves DOWN, we note the current head column and move left until a \# is met. 
Then we move to the previous \# or \$ and make moves right until we are at the column specified from before.
However, if we move left and a \$ is met instead, we return to the original position of the same row
(using the column value) as the table is not expandable downwards. \newline

\noindent When the 2D Turing Machine moves UP, we note the current head column and move right until a \# is met. 
Then we make moves right until we are at the same column specified from before.
However, if we move right and a \$ is met instead, we insert a \# followed by $n$ blanks, \textvisiblespace, 
(where $n$ is the column index) by shifting the \$ $n+1$ cells right.
This is equivalent to adding a new row to the top of the 2D Turing Machine's table. \newline

\noindent As the 2-dimensional Turing Machine can be simulated by a 2-tape standard Turing Machine, 
the 2D Turing Machine is not more powerful than a standard TM.
\end{proof}

\end{document}